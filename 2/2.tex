\documentclass{article}
\usepackage[dutch]{babel}
\usepackage{graphicx}

\usepackage{fontspec}
\setmonofont{Inconsolata}

%\usepackage{tikz}
%\usepackage{tkz-graph}
%\usetikzlibrary{babel,graphdrawing,graphs,arrows.meta,shapes.misc,chains,positioning,shapes,quotes,automata,bending}
%\usegdlibrary{trees}
%\usegdlibrary{layered}

\usepackage[T1]{fontenc}
\usepackage[utf8]{inputenc}
\usepackage{lmodern}
\usepackage[a4paper]{geometry}
%\usepackage{fullpage}
\usepackage{etoolbox}
\usepackage{amsmath}
\usepackage{mathtools}
\usepackage{latexsym}
\usepackage{booktabs}
\usepackage{adjustbox}
\usepackage{minted}


\makeatletter
\patchcmd{\maketitle}{\@fnsymbol}{\@alph}{}{}  % Footnote numbers from symbols to small letters
\makeatother

\title{Huiswerk 2\\ \large{Statistisch Redeneren UvA-2015}}
\author{Jelte Fennema\thanks{Student nummer 10183159} ~\& Bas van den
Heuvel\thanks{Student nummer 10343725}}

\date{\today}

\begin{document}
\maketitle

\begin{enumerate}
    \item We gaan er van uit dat er over een continu interval gesproken wordt.

        \begin{enumerate}
            \item
                $$
                F(x) =
                \begin{dcases*}
                    0 & als $x \leq 3$\\
                    \frac{x-3}6 & als $3 < x \leq 9$\\
                    1 & als $x > 9$\\
                \end{dcases*}
                $$

            \item De kans hierop is 0

            \item De kans hierop is ook 0

        \end{enumerate}

    \item
        \begin{enumerate}
                \newcommand{\kop}{\text{kop}}
                \newcommand{\munt}{\text{munt}}
            \item
                $U = \{\kop, \munt\}$

            \item
                $$
                P(k)= \left. n\choose k\right. p^k(1-p)^{n-k}
                $$

            \item Binomiaal

            \item
                \inputminted{python}{lab2_2_d.py}

                Output:

                \inputminted{text}{lab2_2_d.out}

            tl;dr: De kans is overal 1.

        \end{enumerate}

    \item
        \inputminted{python}{lab2_3.py}
        

\begin{figure}[htbp]
\centering
\includegraphics[width=0.8\textwidth]{/usr/lib64/python3.4/site-packages/tmp/plott.pdf}
\end{figure}



\end{enumerate}

\newpage

\section*{Naive Bayes Classificator}
De classificatie of iemand een man of vrouw is, moet in dit model afhangen van
de lengte, het gewicht en de schoenmaat van de geclassificeerde. Aangenomen is
dat deze maten normaal verdeeld zijn. Om de kansdichtheidsfuncties te vinden van
deze maten, hebben we het gemiddelde en de standaarddeviatie van de dataset
berekend.

De lengte, het gewicht en de schoenmaat zijn de drie gegeven \textit{features}
van de data die geclassificeerd moet worden. Samen vormen zij vector $\vec x =
(x_l,x_g,x_s)$. Gegeven deze features, willen we weten wat de kans is dat de
data hoort bij een man of bij een vrouw ($C_m$ of $C_v$). De kans die het
grootste is wordt als waarheid beschouwd.

Volgens de stelling van Bayes geldt:

$$ p(C_k|\vec x) = \frac{p(C_k)p(\vec x|C_k)}{p(\vec x)}$$

Dit willen we oplossen voor $C_k = C_m$ en $C_k = C_v$.

Omdat voor beide berekeningen geldt dat $p(\vec x)$ hetzelfde is, kunnen we deze
buiten beschouwing laten, dus hoeven we alleen $p(C_k)p(\vec x|C_k)$ op te
lossen.

Volgens de chain rule geldt:

$$p(x_l, x_g, x_s|C_k) = p(x_l|C_k)p(x_g|C_k,x_l)p(x_s|C_k,x_l,x_g)$$

Ervan uitgaande dat de features onafhankelijk zijn, kan worden gesteld dat:

$$p(x_l, x_g, x_s|C_k) = p(x_l|C_k)p(x_g|C_k)p(x_s|C_k)$$

De kansen $p(x_i|C_k)$ zijn te berekenen met de kansdichtheidsfuncties die we
eerder hebben opgesteld.

Wat we uiteindelijk dus moeten bereken is:

$$P(C_m)p(x_l|C_m)p(x_g|C_m)p(x_s|C_m) > P(C_v)p(x_l|C_v)p(x_g|C_v)p(x_s|C_v)$$

Wanneer dat waar is, wordt het te classificeren als man geclassificeerd. Als dit
niet waar is, wordt het geclassificeerd als vrouw.

De code:

\inputminted{python}{naive_noutf.py}

De confusion matrix:

\begin{table}[h]
\begin{tabular}{|l|l|l|}
\hline
                      & daadwerkelijk man & daadwerkelijk vrouw \\ \hline
geclassificeerd man   & 25                & 1                   \\ \hline
geclassificeerd vrouw & 2                 & 5                   \\ \hline
\end{tabular}
\end{table}

\end{document}
