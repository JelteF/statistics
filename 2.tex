\documentclass{article}
\usepackage[dutch]{babel}
\usepackage{graphicx}

\usepackage{tikz}
\usepackage{tkz-graph}
\usetikzlibrary{babel,graphdrawing,graphs,arrows.meta,shapes.misc,chains,positioning,shapes,quotes,automata,bending}
\usegdlibrary{trees}
\usegdlibrary{layered}

\usepackage[utf8]{inputenc}
\usepackage[a4paper]{geometry}
%\usepackage{fullpage}
\usepackage{etoolbox}
\usepackage{amsmath}
\usepackage{mathtools}
\usepackage{latexsym}
\usepackage{booktabs}
\usepackage{adjustbox}
\usepackage{minted}


\makeatletter
\patchcmd{\maketitle}{\@fnsymbol}{\@alph}{}{}  % Footnote numbers from symbols to small letters
\makeatother

\title{Huiswerk 2\\ \large{Statistisch Redeneren UvA-2015}}
\author{Jelte Fennema\thanks{Student nummer 10183159} ~\& Bas van den
Heuvel\thanks{Student nummer 10343725}}

\date{\today}

\begin{document}
\maketitle

\begin{enumerate}
    \item We gaan er van uit dat er over een continu interval gesproken wordt.
        \begin{enumerate}
            \item
                $$
                F(x) =
                \begin{dcases*}
                    0 & als $x \leq 3$\\
                    \frac{x-3}6 & als $3 < x \leq 9$\\
                    1 & als $x > 9$\\
                \end{dcases*}
                $$

            \item De kans hierop is 0

            \item De kans hierop is ook 0

        \end{enumerate}

    \item
        \begin{enumerate}
                \newcommand{\kop}{\text{kop}}
                \newcommand{\munt}{\text{munt}}
            \item
                $U = \{\kop, \munt\}$

            \item
                $$
                P(k)= \left. n\choose k\right. p^k(1-p)^{n-k}
                $$

            \item Binomiaal

            \item
                \begin{minted}{python}
from scipy.misc import comb


def exp(p, n):
    total = 0.0
    for k in range(n+1):
        total += comb(n, k, exact=False) * p**k * (1-p) ** (n-k)

    return total


def main():
    for p in [0.3, 0.75, 0.8, 1.0, 0.0, 0.5]:
        for n in range(1, 20):
            print('Checking n=%d, p=%f' % (n, p))
            print('Result: %f' % (exp(p, n)))

if __name__ == '__main__':
    main()
                \end{minted}

                Output:

                \begin{minted}{text}
Checking n=1, p=0.300000
Result: 1.000000
Checking n=2, p=0.300000
Result: 1.000000
Checking n=3, p=0.300000
Result: 1.000000
Checking n=4, p=0.300000
Result: 1.000000
Checking n=5, p=0.300000
Result: 1.000000
Checking n=6, p=0.300000
Result: 1.000000
Checking n=7, p=0.300000
Result: 1.000000
Checking n=8, p=0.300000
Result: 1.000000
Checking n=9, p=0.300000
Result: 1.000000
Checking n=10, p=0.300000
Result: 1.000000
Checking n=11, p=0.300000
Result: 1.000000
Checking n=12, p=0.300000
Result: 1.000000
Checking n=13, p=0.300000
Result: 1.000000
Checking n=14, p=0.300000
Result: 1.000000
Checking n=15, p=0.300000
Result: 1.000000
Checking n=16, p=0.300000
Result: 1.000000
Checking n=17, p=0.300000
Result: 1.000000
Checking n=18, p=0.300000
Result: 1.000000
Checking n=19, p=0.300000
Result: 1.000000
Checking n=1, p=0.750000
Result: 1.000000
Checking n=2, p=0.750000
Result: 1.000000
Checking n=3, p=0.750000
Result: 1.000000
Checking n=4, p=0.750000
Result: 1.000000
Checking n=5, p=0.750000
Result: 1.000000
Checking n=6, p=0.750000
Result: 1.000000
Checking n=7, p=0.750000
Result: 1.000000
Checking n=8, p=0.750000
Result: 1.000000
Checking n=9, p=0.750000
Result: 1.000000
Checking n=10, p=0.750000
Result: 1.000000
Checking n=11, p=0.750000
Result: 1.000000
Checking n=12, p=0.750000
Result: 1.000000
Checking n=13, p=0.750000
Result: 1.000000
Checking n=14, p=0.750000
Result: 1.000000
Checking n=15, p=0.750000
Result: 1.000000
Checking n=16, p=0.750000
Result: 1.000000
Checking n=17, p=0.750000
Result: 1.000000
Checking n=18, p=0.750000
Result: 1.000000
Checking n=19, p=0.750000
Result: 1.000000
Checking n=1, p=0.800000
Result: 1.000000
Checking n=2, p=0.800000
Result: 1.000000
Checking n=3, p=0.800000
Result: 1.000000
Checking n=4, p=0.800000
Result: 1.000000
Checking n=5, p=0.800000
Result: 1.000000
Checking n=6, p=0.800000
Result: 1.000000
Checking n=7, p=0.800000
Result: 1.000000
Checking n=8, p=0.800000
Result: 1.000000
Checking n=9, p=0.800000
Result: 1.000000
Checking n=10, p=0.800000
Result: 1.000000
Checking n=11, p=0.800000
Result: 1.000000
Checking n=12, p=0.800000
Result: 1.000000
Checking n=13, p=0.800000
Result: 1.000000
Checking n=14, p=0.800000
Result: 1.000000
Checking n=15, p=0.800000
Result: 1.000000
Checking n=16, p=0.800000
Result: 1.000000
Checking n=17, p=0.800000
Result: 1.000000
Checking n=18, p=0.800000
Result: 1.000000
Checking n=19, p=0.800000
Result: 1.000000
Checking n=1, p=1.000000
Result: 1.000000
Checking n=2, p=1.000000
Result: 1.000000
Checking n=3, p=1.000000
Result: 1.000000
Checking n=4, p=1.000000
Result: 1.000000
Checking n=5, p=1.000000
Result: 1.000000
Checking n=6, p=1.000000
Result: 1.000000
Checking n=7, p=1.000000
Result: 1.000000
Checking n=8, p=1.000000
Result: 1.000000
Checking n=9, p=1.000000
Result: 1.000000
Checking n=10, p=1.000000
Result: 1.000000
Checking n=11, p=1.000000
Result: 1.000000
Checking n=12, p=1.000000
Result: 1.000000
Checking n=13, p=1.000000
Result: 1.000000
Checking n=14, p=1.000000
Result: 1.000000
Checking n=15, p=1.000000
Result: 1.000000
Checking n=16, p=1.000000
Result: 1.000000
Checking n=17, p=1.000000
Result: 1.000000
Checking n=18, p=1.000000
Result: 1.000000
Checking n=19, p=1.000000
Result: 1.000000
Checking n=1, p=0.000000
Result: 1.000000
Checking n=2, p=0.000000
Result: 1.000000
Checking n=3, p=0.000000
Result: 1.000000
Checking n=4, p=0.000000
Result: 1.000000
Checking n=5, p=0.000000
Result: 1.000000
Checking n=6, p=0.000000
Result: 1.000000
Checking n=7, p=0.000000
Result: 1.000000
Checking n=8, p=0.000000
Result: 1.000000
Checking n=9, p=0.000000
Result: 1.000000
Checking n=10, p=0.000000
Result: 1.000000
Checking n=11, p=0.000000
Result: 1.000000
Checking n=12, p=0.000000
Result: 1.000000
Checking n=13, p=0.000000
Result: 1.000000
Checking n=14, p=0.000000
Result: 1.000000
Checking n=15, p=0.000000
Result: 1.000000
Checking n=16, p=0.000000
Result: 1.000000
Checking n=17, p=0.000000
Result: 1.000000
Checking n=18, p=0.000000
Result: 1.000000
Checking n=19, p=0.000000
Result: 1.000000
Checking n=1, p=0.500000
Result: 1.000000
Checking n=2, p=0.500000
Result: 1.000000
Checking n=3, p=0.500000
Result: 1.000000
Checking n=4, p=0.500000
Result: 1.000000
Checking n=5, p=0.500000
Result: 1.000000
Checking n=6, p=0.500000
Result: 1.000000
Checking n=7, p=0.500000
Result: 1.000000
Checking n=8, p=0.500000
Result: 1.000000
Checking n=9, p=0.500000
Result: 1.000000
Checking n=10, p=0.500000
Result: 1.000000
Checking n=11, p=0.500000
Result: 1.000000
Checking n=12, p=0.500000
Result: 1.000000
Checking n=13, p=0.500000
Result: 1.000000
Checking n=14, p=0.500000
Result: 1.000000
Checking n=15, p=0.500000
Result: 1.000000
Checking n=16, p=0.500000
Result: 1.000000
Checking n=17, p=0.500000
Result: 1.000000
Checking n=18, p=0.500000
Result: 1.000000
Checking n=19, p=0.500000
Result: 1.000000
            \end{minted}

        \end{enumerate}

\end{enumerate}


\end{document}
