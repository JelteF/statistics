\documentclass{article}
\usepackage[dutch]{babel}
\usepackage{graphicx}

\usepackage{tikz}
\usepackage{tkz-graph}
\usetikzlibrary{babel,graphdrawing,graphs,arrows.meta,shapes.misc,chains,positioning,shapes,quotes,automata,bending}
\usegdlibrary{trees}
\usegdlibrary{layered}

\usepackage[utf8]{inputenc}
\usepackage[a4paper]{geometry}
%\usepackage{fullpage}
\usepackage{etoolbox}
\usepackage{amsmath}
\usepackage{mathtools}
\usepackage{latexsym}
\usepackage{booktabs}
\usepackage{adjustbox}


\makeatletter
\patchcmd{\maketitle}{\@fnsymbol}{\@alph}{}{}  % Footnote numbers from symbols to small letters
\makeatother

\title{Huiswerk 2\\ \large{Statistisch Redeneren UvA-2015}}
\author{Jelte Fennema\thanks{Student nummer 10183159} ~\& Bas van den
Heuvel\thanks{Student nummer 10343725}}

\date{\today}

\begin{document}
\maketitle

\begin{enumerate}
    \item We gaan er van uit dat er over een continu interval gesproken wordt.
        \begin{enumerate}
            \item
                $$
                F(x) =
                \begin{dcases*}
                    0 & als $x \leq 3$\\
                    \frac{x-3}6 & als $3 < x \leq 9$\\
                    1 & als $x > 9$\\
                \end{dcases*}
                $$

            \item De kans hierop is 0

            \item De kans hierop is ook 0

        \end{enumerate}

    \item
        \begin{enumerate}
                \newcommand{\kop}{\text{kop}}
                \newcommand{\munt}{\text{munt}}
            \item
                $U = \{\kop, \munt\}$

            \item
                $$
                P(k)= \left. n\choose k\right. p^k(1-p)^{n-k}
                $$

            \item Binomiaal

            \item

        \end{enumerate}

\end{enumerate}


\end{document}
