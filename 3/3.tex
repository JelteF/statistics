\documentclass{article}
\usepackage[dutch]{babel}
\usepackage{graphicx}

%\usepackage{tikz}
%\usepackage{tkz-graph}
%\usetikzlibrary{babel,graphdrawing,graphs,arrows.meta,shapes.misc,chains,positioning,shapes,quotes,automata,bending}
%\usegdlibrary{trees}
%\usegdlibrary{layered}

\usepackage[T1]{fontenc}
\usepackage[utf8]{inputenc}
\usepackage{lmodern}
\usepackage[a4paper]{geometry}
%\usepackage{fullpage}
\usepackage{etoolbox}
\usepackage{amsmath}
\usepackage{mathtools}
\usepackage{latexsym}
\usepackage{booktabs}
\usepackage{adjustbox}
\usepackage{minted}
\usepackage{enumerate}


\makeatletter
\patchcmd{\maketitle}{\@fnsymbol}{\@alph}{}{}  % Footnote numbers from symbols to small letters
\makeatother

\title{Huiswerk 3\\ \large{Statistisch Redeneren UvA-2015}}
\author{Jelte Fennema\thanks{Student nummer 10183159} ~\& Bas van den
Heuvel\thanks{Student nummer 10343725}}

\date{\today}

\begin{document}
\maketitle

\begin{enumerate}
    \item
        \begin{align}
            p_X(x) &= \sum_{y \in S_Y} p_{XY}(x, y) \\
            E(X+Y) &= \sum_{x \in S_X} \sum_{y \in S_Y} (x+y)p_{XY}(x,y) \notag \\
                   &= \sum_{x \in S_X} \sum_{y \in S_Y} \left(xp_{XY}(x, y)
            +yp_{XY}(x, y) \right) \notag \\
                   &= \sum_{x \in S_X} \left(xp_X(x) + \sum_{y \in S_Y}
            yp_{XY}(x,y)\right)
            & (1) \notag \\
                   &= \sum_{x \in S_X} xp_X(x) + \sum_{y \in S_Y} yp_Y(y) & (1)
            \notag \\
                   &= E(X) + E(Y) \notag \\
                   & \Box \notag
        \end{align}

    \item
        \begin{enumerate}[1.]
            \item Zie Figuur 1
                \input{np_random.tex}



            \item  Zie Figuur 2
                \input{ibm_random.tex}

            \item
                \newcommand{\modd}{\bmod 2^{31}}
                \begin{align*}
                    x_{k+2} &= ax_{k+1}  \modd \\
                            &= a(ax_k\modd)  \modd\\
                            &= a(ax_k)  \modd\\
                            &= a^2x_k  \modd\\
                            &= (2^{16} + 3)^2x_k  \modd\\
                            &= (2^{16} + 3)(2^{16} + 3)x_k  \modd\\
                            &= (2^{16} + 3)(2^{16} + 3)x_k  \modd\\
                            &= 2^{32}x_k + 6\times2^{16}x_k + 9x_k  \modd\\
                            &= 2\times2^{31}x_k + 6\times2^{16}x_k + 9x_k  \modd\\
                            &= 6\times2^{16}x_k + 9x_k  \modd\\
                            &= 6\times2^{16}x_k + 9x_k  \modd\\
                            &= 6\times2^{16}x_k + 18x_k - 9xk  \modd\\
                            &= 6(2^{16}x_k + 3x_k) - 9xk  \modd\\
                            &= 6((2^{16} + 3)x_k) - 9xk  \modd\\
                            &= 6((2^{16} + 3)x_k \modd) - 9xk  \modd\\
                            &= 6x_{k+1} - 9xk  \modd\\
                \end{align*}

            \item
                Numpy gebruikt de Mersenne Twister P(seudo)RNG. Het voordeel
                hiervan is dat alle bits van de getallen die deze PRNG
                produceert evenveel willekeurig zijn. Dit is bij de LCG niet het
                geval. Bovendien is de periode van Mersenne Twister enorm
                ($2^{19937}-1$), waardoor getallen zich niet snel zullen
                herhalen.

        \end{enumerate}

    \item
        \begin{enumerate}[1.]
            \item
                $$E(X) = \frac1{\lambda}$$

            \item
                Het is de snelheid of intensiteit van de verdeling.

            \item
                $$\widehat{\lambda}=\frac1{\overline{x}}$$

            \item
                \inputminted{python}{expo.py}

            \item
                Zie figuur 3.
                \input{expo.tex}

            \item De Python code print de geschatte $\lambda$.

        \end{enumerate}

\end{enumerate}

\end{document}
